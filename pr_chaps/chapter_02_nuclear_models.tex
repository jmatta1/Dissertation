%
%  Chapter:  2 - Nuclear Models for High Spin Phenomena
%  Modified: 2/16/2015
%  Author:   James Till Matta
%
%%%%%%%%%%%%%%%%%%%%%%%%%%%%%%%%%%%%%%%%%%%%%%%%%%%%%%%%%%

\chapter{NUCLEAR MODELS FOR HIGH SPIN PHENOMENA}
\label{chp:models}
The atomic nucleus, discovered in $1911$ by Ernest Rutherford\cite{rutherfordNuclearModel}, is a tiny point of matter at the heart of an atom. This point of matter, is approximately $1-10fm$ across, contains more than $99.94\%$ of an atom's mass, and is composed of protons and neutrons. Since its discovery the nucleus has been studied and characterized using ever more sophisticated models.

By observation of the masses (and by extension particle separation energies)
\section{Introduction}
\label{sec:models-into}
\section{The Shell Model}
\label{sec:models-shell-model}
\subsection{The Deformed Shell Model}
\label{ssec:models-shell-model-def-sm}
\section{Rigid Rotor Model}
\label{sec:models-rigid-rotor}

\section{Tilted Axis Cranking}
\label{sec:models-tac}

\section{Wobbling Vibrations in Nuclei}
\label{sec:models-wobbling}
\subsection{Quasiparticle Triaxial Rotor (QTR)}
\label{sec:models-qtr}
\subsection{Signatures of Wobbling}
\label{sec:models-sig}
