%
%  Chapter:  5 - Wobbling Discussion
%  Modified: 2/16/2015
%  Author:   James Till Matta
%
%%%%%%%%%%%%%%%%%%%%%%%%%%%%%%%%%%%%%%%%%%%%%%%%%%%%%%%%%%

\chapter{SUMMARY AND OUTLOOK}
\label{chp:wob-disc}

The purpose of this work was two-fold: confirm the theory of transverse wobbling by confirming its existence in \pr{} and show that wobbling can be found outside the $A\sim{}170$ mass region. These objectives have been met.

By rearranging the negative parity level scheme and performing a careful angular distribution analysis of the $n_w=1\rightarrow{}n_w=0$ transitions in that level scheme, clear evidence for the existence of the transverse wobbling mode in \pr{} was found. The $\Delta{}I=1, E2$ nature of the transitions gives clear evidence for wobbling. Additionally, the decreasing wobbling energy confirms that the wobbling observed was in fact transverse in nature. This confirms Frauendorf's and D\"onau's theory and suggests that the wobblers seen in the $A\sim{}170$ region in Refs. \cite{wobblingIn163Lu,wobblingIn163LuTwoPhonon,wobblingIn165Lu,wobblingIn167Lu,wobblingIn161Lu,wobblingIn167Ta} should be reevaluated as having transverse nature as well.

The experimental data for the zero and one phonon wobbling states of this nucleus are in fair agreement with QTR calculations. The wobbling energy of the band exhibits the characteristic decrease of a transverse wobbler; however at $J^{\pi}=\sfrac{29}{2}^-$ a minimum is reached and the wobbling energy starts to increase. Calculations show that this is due to the Coriolis force realigning the angular momentum, $\vec{j}$, of the $h_{11/2}$ proton from the short to the medium axis, causing a transition from the transverse wobbling mode to the longitudinal mode. The experimental upturn is more pronounced due to the yrast band transitioning from a single quasiparticle $\pi{}(h_{\sfrac{11}{2}})$ structure to a $\pi{}(h_{\sfrac{11}{2}})^3\nu{}(h_{\sfrac{11}{2}})^2$ five quasiparticle structure. Intensity in the wobbling band disappears above $J^{\pi}=\sfrac{33}{2}^-$ due to the appearance of an energetically favorable dipole band with configuration $\pi{}(h_{\sfrac{11}{2}})\nu{}(h_{\sfrac{11}{2}})^2$ whose band-head has the same $J^{\pi}$ and nearly degenerate energy.

A possible $n_w=2$ band is observed in this work. A further high statistics measurement is needed to confirm the nature of this band. Should the band indeed be a two phonon wobbling band then \pr{} would be the third nucleus to exhibit this, after $^{163}$Lu and $^{165}$Lu. A proposal for this experiment has been submitted to the ATLAS program advisory committee and beam time has been granted.

With the observation of transverse wobbling in \pr{}, the $A\sim{}130$ region has been opened for searches for wobbling. While this region is known to have triaxial deformation, confirmed by the many observations of chirality in the region \cite{chiralityIn134Pr,chiralityA130Region,chiralityUpperA130Region,chiralityA130Region2,chirality136Pm,chiralityMore135Nd,chiralityIn135Nd,chiralityMulti133Cs}, this is the first observation of wobbling in this region. Studies in this region following up on the work in Ref. \cite{mattaTransversePRL} show longitudinal wobbling in $^{133}$La \cite{palitLongWobbling}.

